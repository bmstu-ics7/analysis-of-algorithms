\newpage
\section{Технологическая часть}

Необходимо выбрать средства для разработки алгоритмов и подготовить
тесты.

\subsection{Требования к программному обеспечению}

Программное обеспечение должно обеспечивать замер процессорного времени
выполнения каждого алгоритма. Проводятся замеры для случайно генерируемых
строк размерности до 1000.

\subsection{Средства реализации}

В качестве языка программирования был выбран C++.
Данный язык имеет высокую скорость и богатую стандартную библиотеку,
содержащую необходимые контейнеры для данной работы. Программа, написанная на C++,
будет доступна на всех платформах.

Время замерялось с помощью библиотеки chrono, которая измеряет процессорное время \cite{chrono}.

\subsection{Листинг кода}

Результаты разработки указаны на листингах 1-3.

\begin{lstlisting}[caption=Расстояние Левенштейна матричный метод]
Matrix Lmatrix::find(std::string s1, std::string s2)
{
    Matrix matrix;
    for (int i = 0; i < s1.size() + 1; ++i) {
        matrix.push_back(std::vector< int >(s2.size() + 1));
    }

    for (int i = 0; i < matrix.size(); ++i) {
        matrix[i][0] = i;
    }

    for (int j = 0; j < matrix[0].size(); ++j) {
        matrix[0][j] = j;
    }

    for (int i = 1; i < matrix.size(); ++i) {
        for (int j = 1; j < matrix[0].size(); ++j) {
            int u = matrix[i - 1][j] + 1;
            int l = matrix[i][j - 1] + 1;
            int lu = matrix[i - 1][j - 1];
            if (s1[i - 1] != s2[j - 1]) lu++;

            matrix[i][j] = std::min(std::min(u, l), lu);
        }
    }

    return matrix;
}
\end{lstlisting}

\begin{lstlisting}[caption=Расстояние Дамерау-Левенштейна матричный метод]
Matrix DLmatrix::find(std::string s1, std::string s2)
{
    Matrix matrix;
    for (int i = 0; i < s1.size() + 1; ++i) {
        matrix.push_back(std::vector< int >(s2.size() + 1));
    }

    for (int i = 0; i < matrix.size(); ++i) {
        matrix[i][0] = i;
    }

    for (int j = 0; j < matrix[0].size(); ++j) {
        matrix[0][j] = j;
    }

    int inf = std::max(s1.size(), s2.size()) + 1;

    for (int i = 1; i < matrix.size(); ++i) {
        for (int j = 1; j < matrix[0].size(); ++j) {
            int u = matrix[i - 1][j] + 1;
            int l = matrix[i][j - 1] + 1;
            int lu = matrix[i - 1][j - 1];
            int lluu = inf;
            if (i > 1 && j > 1) lluu = matrix[i - 2][j - 2];

            if (s1[i - 1] != s2[j - 1]) lu++;
            if (i > 1 && j > 1 &&
                s1[i - 1] == s2[j - 2] &&
                s1[i - 2] == s2[j - 1])
                lluu++;
            else lluu = inf;

            matrix[i][j] = std::min(std::min(u, l),
                                    std::min(lu, lluu));
        }
    }

    return matrix;
}
\end{lstlisting}

\begin{lstlisting}[caption=Расстояние Дамерау-Левенштейна рекурсивный метод]
int DLrecursive::find(std::string s1, std::string s2)
{
    if (s1.size() == 0) return s2.size();
    if (s2.size() == 0) return s1.size();

    return std::min(
            std::min(
                find(s1.substr(0, s1.size() - 1), s2) + 1,
                find(s1, s2.substr(0, s2.size() - 1)) + 1
            ),
            std::min(
                find(s1.substr(0, s1.size() - 1),
                        s2.substr(0, s2.size() - 1)) +
                (s1[s1.size() - 1] == s2[s2.size() - 1] ? 0 : 1),

                (s1.size() > 1 &&
                s2.size() > 1 &&
                s1[s1.size() - 2] == s2[s2.size() - 1] &&
                s1[s1.size() - 1] == s2[s2.size() - 2] ?
                find(s1.substr(0, s1.size() - 2),
                        s2.substr(0, s2.size() - 2)) :
                    int(std::max(s1.size(), s2.size()))) + 1
            )
        );
}

\end{lstlisting}

\subsection{Тестирование}

Для тестирования программы были заготовлены следующие тесты

\hfill

\begin{flushright}
    Таблица 1
\end{flushright}

\begin{center}
    Тесты для алгоритма Левенштейна

    \begin{tabular}{|c|c|c|}
        \hline
        $s_1$ & $s_2$ & Ожидаемый результат \\
        \hline
        word & word & 0 \\
        \hline
        word & another & 6 \\
        \hline
        ab & ba & 2 \\
        \hline
        qwerty & qwetry & 2 \\
        \hline
        abcdef & badcfe & 4 \\
        \hline
        werylongword & sh & 12 \\
        \hline
        sh & werylongword & 12 \\
        \hline
        wednesday & weekend & 5 \\
        \hline
        memory & mem & 3 \\
        \hline
        feature & erutaef & 6 \\
        \hline
    \end{tabular}
\end{center}

\hfill

\begin{flushright}
    Таблица 2
\end{flushright}

\begin{center}
    Тесты для алгоритма Дамерау-Левенштейна

    \begin{tabular}{|c|c|c|}
        \hline
        $s_1$ & $s_2$ & Ожидаемый результат \\
        \hline
        word & word & 0 \\
        \hline
        word & another & 6 \\
        \hline
        ab & ba & 1 \\
        \hline
        qwerty & qwetry & 1 \\
        \hline
        abcdef & badcfe & 3 \\
        \hline
        werylongword & sh & 12 \\
        \hline
        sh & werylongword & 12 \\
        \hline
        wednesday & weekend & 5 \\
        \hline
        memory & mem & 3 \\
        \hline
        feature & erutaef & 5 \\
        \hline
    \end{tabular}
\end{center}

Все тесты выполнены успешно.

\subsection{Выводы}

Для сравнения были реализованы 3 алгоритма на выбранном языке
программирования C++. Чтобы проверить правильность работы алгоритмов
были подготовлены тесты.

