\newpage
\section{Технологическая часть}

Стоит задача реализовать алгоритмы поиска подстроки в строке.

\subsection{Средства реализации}

В качестве языка программирования был выбран {\ttfamily C++}.
Данный язык имеет высокую скорость и богатую стандартную библиотеку,
содержащую необходимые контейнеры для данной работы. Программа, написанная на
{\ttfamily C++}, будет доступна на всех платформах.

\subsection{Листинг кода}

На листингах \ref{lst:kmp} и \ref{lst:bm} предсавлены алгоритмы
Кнута-Морриса-Пратта и Бойлера-Мура. На листинге \ref{lst:fail} представлен
алгоритм создания массива сдвигов для алгоритма Кнута-Морриса-Пратта.

\begin{lstlisting}[caption=Алгоритм Кнута-Морриса-Пратта,label=lst:kmp]
int KnuthMorrisPratt::find(const std::string& str,
                           const std::string& substr)
{
    if (substr.size() > str.size()) return -1;

    std::vector< int > fail = makeFail(substr);

    int sub = 0;
    int txt = 0;

    while (txt < int(str.size()) && sub < int(substr.size())) {
        if (sub == -1 || str[txt] == substr[sub]) {
            ++txt;
            ++sub;
        } else {
            sub = fail[sub];
        }
    }

    if (sub >= substr.size())
        return txt - substr.size();

    return -1;
}
\end{lstlisting}

\begin{lstlisting}[caption=Алгоритм создания массива сдвигов,label=lst:fail]
std::vector< int > KnuthMorrisPratt::makeFail(
    const std::string& substr
)
{
    std::vector< int > fail(substr.size());

    fail[0] = -1;

    for (int i = 1; i < fail.size(); ++i) {
        int temp = fail[i - 1];

        while (temp > -1 && substr[temp] != substr[i - 1]) {
            temp = fail[temp];
        }

        fail[i] = temp + 1;
    }

    return fail;
}
\end{lstlisting}

\begin{lstlisting}[caption=Алгоритм Бойлера-Мура,label=lst:bm]
int BoilerMyr::find(const std::string& str, const std::string& substr)
{
    if (substr.size() > str.size()) return -1;

    std::vector< int > offset;

    for (int i = 0; i <= 255; ++i) {
        offset.push_back(int(substr.size()));
    }

    for (int i = 0; i < substr.size() - 1; ++i) {
        offset[int(substr[i])] = substr.size() - i - 1;
    }

    int i = int(substr.size()) - 1;
    int txt = i;
    int sub = i;

    while (txt > 0 && i < int(str.size())) {
        txt = int(substr.size()) - 1;
        sub = i;

        while (txt >= 0 && str[sub] == substr[txt]) {
            --sub;
            --txt;
        }

        i += offset[int(str[i])];
    }

    if (sub > int(str.size()) - int(substr.size())) return -1;

    return sub + 1;
}
\end{lstlisting}

\subsection{Тестирование}

Для тестирования были заготовлены следующие тесты, которые представлены
на таблице \ref{table:test}.

\begin{table}[H]
    \caption{Тесты}
    \label{table:test}
    \centering
    \begin{tabular}{|c|c||c|}
        \hline
        Строка & Подстрока & Ожидаемый результат \\
        \hline
        \hline
        abababbc & ababbc & 2 \\
        \hline
        qwe & qwe & 0 \\
        \hline
        asdasd & sa & -1 \\
        \hline
        abababa & baba & 1 \\
        \hline
    \end{tabular}
\end{table}

\subsection{Выводы}

Были реализованы алгоритмы поиска подстроки в строке и подготовлены тесты.
